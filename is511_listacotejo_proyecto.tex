\documentclass[12pt]{article}
\usepackage[utf8]{inputenc}
\usepackage[spanish]{babel}
%\usepackage[spanish]{babel}
%\usepackage[T1]{fontenc}
%\usepackage{dejavu}
\usepackage{enumitem}
\usepackage[table,xcdraw]{xcolor}

\usepackage{listings}
\usepackage{amsfonts}
\usepackage{fancyhdr}
\usepackage{comment}
\usepackage{graphicx}
\usepackage[letterpaper, top=2.5cm, bottom=2.5cm, left=2.2cm, right=2.2cm]{geometry}
%\renewcommand{\item}[1]{\item \textbf{#1}}
\begin{document}

\begin{center}
%\includegraphics{logo_unah.png}\\
\bfseries{Universidad Nacional Autónoma de Honduras}\\
Facultad de Ingeniería\\
Departamento de Ingeniería en Sistemas\\
\bigskip
\bigskip
IS-511 Redes de Datos 1 1500\\
II PAC 2018
\end{center}
\begin{center}
\noindent\rule{\textwidth}{1pt}
{\huge\textbf Diseño y documentación de una red de datos}\\
\vspace{10px}
Lista de cotejo de Proyecto de Asignatura
\noindent\rule{\textwidth}{1pt}
\end{center}

\section{Temas a evaluar en el desarrollo del proyecto}
\begin{enumerate}
\item Direccionamiento en IPv4
\item DHCP
\item Configuración del protocolo de enrutamiento RIP
\item Configuración de comandos básicos en dispositivos Cisco:
	\begin{itemize}
    	\item Hostname
    	\item Banner motd
        \item Contraseña de líneas VTY
        \item Contraseña de consola
        \item Contraseña enable
        \item Encriptamiento de contraseñas
	        \item logging synchronous.
        \item Búsqueda de comandos en dominio (ip domain-lookup)
    \end{itemize}
\end{enumerate}

\section{Evaluación del proyecto}
\begin{enumerate}
\item ¿el estudiante presenta el esquema de direccionamiento IPv4 LAN y WAN? (1 punto)	sí	no
\item ¿el estudiante presenta el diagrama de topología?	(1 punto) sí	no
\item ¿el estudiante presenta la documentación de equipo utilizado?(1 punto)	sí	no
\item ¿ el estudiante presenta la documentación de contraseñas de los dispositivos?(1 punto)	sí	no
\item ¿ el estudiante presenta la documentación de DHCP?(1 punto)	sí	no
\item ¿se presenta la implementación de la topología requerida y ésta cumple con los requerimientos solicitados? (14 puntos)
\item ¿la topología tiene complemente configurado el direccionamiento IPv4? (5 puntos)
\item ¿se ha configurado el DHCP en los routers que era necesario? (1 punto)
\item ¿se ha realizado la configuración de los parámetros básicos? (1 punto)
\item ¿se ha configurado el protocolo RIP? (1 punto)
\item ¿la conectividad entre hosts de distintas empresas es exitosa? (1 punto)
\item ¿se responden [correctamente] las siguientes preguntas? (7 puntos)
\begin{enumerate}
\item ¿cuáles son las topologías LAN y WAN que identifica en su diseño?
\item ¿cuál es la utilidad de configurar DHCP?
\item ¿qué capa de TCP/IP es la encargada de gestionar el direccionamiento?
\item Haga una lista de recomendaciones para futuros estudiantes que realizarán un proyecto de asignatura similar.
\item Haga una lista de aspectos que le llamaron la atención de trabajar el proyecto de asignatura
\item Haga una lista de aspectos que considera que debe reforzar en relación a la asignatura de Redes de Datos I
\item Haga una votación en su equipo, ¿cuál fue el integrante más valioso para el desarrollo del proyecto?
\end{enumerate}
\end{enumerate}

Los estudiantes deben colocar su nombre en este espacio seguido de su firma. Al firmar, los estudiantes aceptan la calificación asignada.
%\newpage
\vfill
{\textbf {\normalsize José Mario López}}

{\small Profesor de la asignatura\\}

{\footnotesize 

\end{document}