\documentclass[12pt]{article}
\usepackage[utf8]{inputenc}
\usepackage[spanish]{babel}
%\usepackage[spanish]{babel}
%\usepackage[T1]{fontenc}
%\usepackage{dejavu}
\usepackage{enumitem}
\usepackage[table,xcdraw]{xcolor}

\usepackage{listings}
\usepackage{amsfonts}
\usepackage{fancyhdr}
\usepackage{comment}
\usepackage{graphicx}
\usepackage[letterpaper, top=2.5cm, bottom=2.5cm, left=2.2cm, right=2.2cm]{geometry}
%\renewcommand{\item}[1]{\item \textbf{#1}}
\begin{document}

\begin{center}
%\includegraphics{logo_unah.png}\\
\bfseries{Universidad Nacional Autónoma de Honduras}\\
Facultad de Ingeniería\\
Departamento de Ingeniería en Sistemas\\
\bigskip
\bigskip
Asignatura: IS-611 Redes de Datos 2\\
Impartida  por José Mario López
\end{center}
\begin{center}
\noindent\rule{\textwidth}{1pt}
{\huge\textbf Diseño y documentación de una red de datos}\\
\vspace{10px}
Proyecto de Asignatura
\noindent\rule{\textwidth}{1pt}
\end{center}

\section{Introducción} 
El objetivo del desarrollo del proyecto es afianzar habiliades en la configuración y comprensión del funcionamiento de protocolos de enrutamiento dinámico, gestión de VLANs y características detalladas de una red local como agregación de enlaces, seguridad lógica en dispositivos de red y redundancia de gateway. Además, el proyecto propicia que el estudiante conozca y seleccione equipos de red para utilizarlos en una topología que cumpla con los requerimientos presentados, y gestionar una documentación adecuada de la misma.

\section{Temas a conocer para el desarrollo del proyecto}
\begin{enumerate}
\item Direccionamiento en IPv4
\item DHCP
\item VLANs y VTP
\item Spanning Tree Protocol (STP)
\item Listas de Control de Acceso (ACL)
\item Redundancia física de enlaces en Capa 2 y Capa 3
\item Configuración de redundancia de gateway con protocolos HSRP y GLBP.
\item Enrutamiento dinámico con RIPv2, EIGRP y OSPF
\item Seguridad de puertos de switch
\item Configuración de comandos básicos en dispositivos Cisco:
	\begin{itemize}
    	\item Hostname
    	\item Banner motd
        \item Habilitación de SSH a través de líneas VTY
        \item Contraseña de consola
        \item Contraseña enable
        \item Encriptamiento de contraseñas
        \item logging synchronous.
    \end{itemize}
\end{enumerate}

\section{Descripción del proyecto}
\subsection{Direccionamiento en IPv4}
\subsubsection{Requerimientos de direccionamiento para usuarios}
Se pide diseñar una topología de red que cumpla con los siguientes requerimientos de direccionamiento:

\textbf{Empresa \#1 Inversiones La Paz, S.A.}
\begin{enumerate}
\item 95 usuarios del departamento de Ventas
\item 13 usuarios del departamento de Compras
\item 14 usuarios del departamento de Dirección
\end{enumerate}

\textbf{Empresa \#2 Aire Frio}
\begin{enumerate}
\item 20 usuarios en el área de técnicos
\item 5 usuarios del área de administración
\item 2 usuarios del área de dirección
\end{enumerate}

\textbf{Empresa \#3 Consultores Emprendo}
\begin{enumerate}
\item 60 usuarios en el área de capacitación
\item 5 usuarios del área de gestión
\item 10 usuarios del área de invitados (acceso inlámbrico)
\end{enumerate}

\subsubsection{Tareas a realizar}
\begin{enumerate}
\item Definir estructura de direccionamiento, teniendo en cuenta que el espacio de direcciones es 172.16.0.0/16 para la empresa\#1, 192.168.10.0/25 para la empresa\#2, y 192.168.11.0/25 para la empresa\#3. Para conexiones WAN use bloques /30, a partir de 10.0.0.0
\item Diseñar topología de red que cumpla con redundancia física en capa 2 y capa 3.
%\item Implementar la topología diseñada en Packet Tracer v. 7.1
\item Documentar la asignación de direcciones IPv4 utilizando los ANEXOS 1 y 2.
\end{enumerate}

\subsection{Implementación de la topología}
\subsubsection{Packet Tracer v. 7.1}
\begin{enumerate}
\item Realizar la conexión de la topología en Packet Tracer. Utilice el ANEXO 3 para el registro de los dispositivos.
\item Configurar el direccionamiento IPv4
\item Realizar la configuración de los parámetros básicos. Para el hostname utilice R o S, según corresponda seguido de un número, e.g. R1. Para el registro de contraseñas utilice el ANEXO 4.
\end{enumerate}

\subsubsection{Configuración de VLANs}
\begin{enumerate}
\item Habilite VTP. Registre el nombre de dominio y el password utilizado.
\item Cree las VLANs necesarias y haga el registro del VLAN ID, el nombre asignado y el propósito.
\end{enumerate}

\subsubsection{Configuración de enrutamiento dinámico}
\begin{enumerate}
\item Los routers de la empresa\#1 anuncian con EIGRP. Utilice la wildcard correspondiente. AS 101.
\item Los routers de la empresa\#2 anuncian con OSPF. ID 100.
\item Los routers de la empresa\#3 anuncian con EIGRP. AS 102
\item Realice la redistribución de información de enrutamiento en los routers que corresponda.
\item Haga un registro del protocolo que utiliza cada dispositivo, con su respectivo AS o ID según corresponda. Utilice los ANEXOS para tomar una guía.
\end{enumerate}

\subsection{Seguridad de los equipos}
\begin{enumerate}
\item Habilite seguridad en los switches para que los puertos aprendan un maximo de 5 direcciones MAC.
\item Habilite las ACL respectivas (se enunciarán a posterior)
\end{enumerate}

\section{Presentación}
Debe entregar un archivo de Packet Tracer con el proyecto desarrollado. Adicional, debe presentar un documento con la documentación (ANEXOS solicitados).
El documento debe tener la siguiente estructura:
\begin{enumerate}
\item Portada
\item Breve descripción del proyecto
\item Índice de anexos
\item Anexos
\item Hoja que contenga la autoevaluación de cada miembro del grupo. (guiarse de la pregunta: ¿De 0-5 qué calificación merece su desempeño en la realización del proyecto?)
\end{enumerate}

Las hojas del documento deben ir \textbf{grapadas}. No presente folder, sobre de papel manila o cualquier extra.
El documento debe ir impreso a doble cara.


%%%%
\newpage
\section{ANEXO 1}

\noindent Nombre de empresa:\\
Espacio de direccionamiento:\\

\begin{table}[ht]
\centering
\caption{Registro de direccionamiento IPv4 -  LAN}
\begin{tabular}{|p{0.10\linewidth}|p{0.10\linewidth}|p{0.10\linewidth}|p{0.10\linewidth}|p{0.10\linewidth}|p{0.10\linewidth}|p{0.10\linewidth}|}
\hline
\rowcolor[HTML]{C0C0C0} 
\# subred & Dirección de red & Dirección de broadcast & Gateway & Máscara de subred & Espacio utilizado & Espacio libre            \\ \hline
          &                  &                        &         &                    &                   & \\ \hline
          &                  &                        &         &                    &                   & \\ \hline
          &                  &                        &         &                    &                   & \\ \hline
          &                  &                        &         &                    &                   & \\ \hline
\end{tabular}
\end{table}
%%%%

\section{ANEXO 2}

\begin{table}[ht]
\centering
\caption{Registro de direccionamiento IPv4 - WAN}
\begin{tabular}{|p{0.10\linewidth}|p{0.10\linewidth}|p{0.10\linewidth}|p{0.10\linewidth}|p{0.10\linewidth}|p{0.10\linewidth}|p{0.10\linewidth}|}
\hline
\rowcolor[HTML]{C0C0C0} 
\# subred & Router extremo 1 & Dirección IP & Router extremo 2 & Dirección IP\\ \hline
          &                  &                        &         &                     \\ \hline
          &                  &                        &         &                     \\ \hline
          &                  &                        &         &                    \\ \hline
          &                  &                        &         &                    \\ \hline
\end{tabular}
\end{table}

\section{ANEXO 3}
\begin{table}[ht]
\centering
\caption{Registro de equipo utilizado}
\begin{tabular}{|p{0.10\linewidth}|p{0.10\linewidth}|p{0.10\linewidth}|p{0.10\linewidth}|p{0.10\linewidth}|p{0.10\linewidth}|p{0.10\linewidth}|}
\hline
\rowcolor[HTML]{C0C0C0} 
Cantidad & Tipo de equipo & Modelo & \# y tipo de interfaces\\ \hline
          &                  &                        &               \\ \hline
          &                  &                        &                    \\ \hline
          &                  &                        &                   \\ \hline
          &                  &                        &                   \\ \hline
\end{tabular}
\end{table}
* tipo de equipo: router o switch

\section{ANEXO 4}
\begin{table}[ht]
\centering
\caption{Registro de contraseñas}
\begin{tabular}{|p{0.10\linewidth}|p{0.10\linewidth}|p{0.10\linewidth}|p{0.10\linewidth}|p{0.10\linewidth}|p{0.10\linewidth}|p{0.10\linewidth}|}
\hline
\rowcolor[HTML]{C0C0C0} 
Hostname & enable & línea de consola & línea VTY\\ \hline
          &                  &                        &               \\ \hline
          &                  &                        &                    \\ \hline
          &                  &                        &                   \\ \hline
          &                  &                        &                   \\ \hline
\end{tabular}
\end{table}
%%%%
%\newpage
\vfill
{\textbf {\normalsize José Mario López}}

{\small Profesor de la asignatura\\}

{\footnotesize 
15 de agosto de 2018
}

\end{document}