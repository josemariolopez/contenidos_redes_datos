\documentclass[12pt]{article}
\usepackage[utf8]{inputenc}
\usepackage[spanish]{babel}
%\usepackage[spanish]{babel}
%\usepackage[T1]{fontenc}
%\usepackage{dejavu}
\usepackage{enumitem}
\usepackage[table,xcdraw]{xcolor}

\usepackage{listings}
\usepackage{amsfonts}
\usepackage{fancyhdr}
\usepackage{comment}
\usepackage{graphicx}
\usepackage[letterpaper, top=2.5cm, bottom=2.5cm, left=2.2cm, right=2.2cm]{geometry}
%\renewcommand{\item}[1]{\item \textbf{#1}}
\begin{document}

\begin{center}
%\includegraphics{logo_unah.png}\\
\bfseries{Universidad Nacional Autónoma de Honduras}\\
Facultad de Ingeniería\\
Departamento de Ingeniería en Sistemas\\
\bigskip
\bigskip
Asignatura: IS-611 Redes de Datos 2\\
Impartida  por José Mario López
\end{center}
\begin{center}
\noindent\rule{\textwidth}{1pt}
{\huge\textbf Diseño y documentación de una red de datos}\\
\vspace{10px}
Lista de cotejo de Proyecto de Asignatura
\noindent\rule{\textwidth}{1pt}
\end{center}

\section{Temas a evaluar en el desarrollo del proyecto}
\begin{enumerate}
\item Direccionamiento en IPv4
\item DHCP
\item VLANs y VTP
\item Spanning Tree Protocol (STP)
\item Listas de Control de Acceso (ACL)
\item Redundancia física de enlaces en Capa 2 y Capa 3
\item Configuración de redundancia de gateway con protocolos HSRP y GLBP.
\item Enrutamiento dinámico con RIPv2, EIGRP y OSPF
\item Seguridad de puertos de switch
\item Configuración de comandos básicos en dispositivos Cisco:
	\begin{itemize}
    	\item Hostname
    	\item Banner motd
        \item Habilitación de SSH a través de líneas VTY
        \item Contraseña de consola
        \item Contraseña enable
        \item Encriptamiento de contraseñas
        \item logging synchronous.
    \end{itemize}
\end{enumerate}

\section{Evaluación del proyecto}
\begin{enumerate}
\item ¿el estudiante presenta el esquema de direccionamiento IPv4 LAN y WAN? (1 punto)	sí	no
\item ¿el estudiante presenta el diagrama de topología?	(1 punto) sí	no
\item ¿el estudiante presenta la documentación de equipo utilizado?(1 punto)	sí	no
\item ¿ el estudiante presenta la documentación de contraseñas de los dispositivos?(1 punto)	sí	no
\item ¿ el estudiante presenta la documentación de DHCP?(1 punto)	sí	no
\item ¿el diseño de la topología cumple con redundancia física LAN y WAN? (5 puntos)
\item ¿se ha configurado redundancia de gateway? (2 puntos)	sí	no
\item ¿se ha configurado protección de puertos en los switches? (2 puntos)	sí	no
\item ¿se ha configurado VLANS y VTP, y se presenta la documentación? (2 puntos)
\item ¿la topología tiene complemente configurado el direccionamiento IPv4? (2 puntos)
\item ¿se ha configurado el DHCP en los routers que era necesario? (1 punto)
\item ¿se ha realizado la configuración de los parámetros básicos? (2 puntos)
\item ¿se han configurado los protocolos de enrutamiento dinámico, redistribución y presenta documentación? (4 punto)
\item ¿la conectividad entre hosts de distintas empresas es exitosa? (2 puntos)
\item ¿se responden [correctamente] las siguientes preguntas? (6 puntos)
\begin{enumerate}
\item ¿cuáles son las topologías LAN y WAN que identifica en su diseño?
\item ¿cuál es la utilidad de implementar STP en la topología?
\item ¿Cuál protocolo de enrutamiento es mejor para utilizar en la topología, EIGRP u OSPF?
\item ¿El equipo ha configurado correctamente la ACL solicitada?
\item Haga una lista de recomendaciones para futuros estudiantes que realizarán un proyecto de asignatura similar.
\item Haga una lista de aspectos que le llamaron la atención de trabajar el proyecto de asignatura
\item Haga una lista de aspectos que considera que debe reforzar en relación a la asignatura de Redes de Datos I
\item Haga una votación en su equipo, ¿cuál fue el integrante más valioso para el desarrollo del proyecto?
\end{enumerate}
\end{enumerate}

Los estudiantes deben colocar su nombre en este espacio seguido de su firma. Al firmar, los estudiantes aceptan la calificación asignada.
%\newpage
\vfill
{\textbf {\normalsize José Mario López}}

{\small Profesor de la asignatura\\}

{\footnotesize 

\end{document}