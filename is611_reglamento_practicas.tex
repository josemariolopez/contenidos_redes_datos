\documentclass[12pt]{article}
\usepackage[utf8]{inputenc}
%\usepackage[T1]{fontenc}
%\usepackage{dejavu}

\usepackage{listings}
\usepackage{amsfonts}
\usepackage{fancyhdr}
\usepackage{comment}
\usepackage{graphicx}
\usepackage[letterpaper, top=2.5cm, bottom=2.5cm, left=2.2cm, right=2.2cm]{geometry}
\renewcommand{\theenumi}{\Alph{enumi}}
\usepackage{enumitem}


\begin{document}
\begin{center}
%\includegraphics{logo_unah.png}\\
\bfseries{Universidad Nacional Autónoma de Honduras}\\
Facultad de Ingeniería\\
Departamento de Ingeniería en Sistemas\\
\bigskip
\bigskip
Asignatura: IS-611 Redes de Datos 2\\
Impartida  por José Mario López
\end{center}
\begin{center}


\noindent\rule{\textwidth}{1pt}
\huge{DE LAS PRÁCTICAS DE CLASE}
\noindent\rule{\textwidth}{1pt}
\end{center}
%\title{Resumen sobre Spanning-Tree Protocol}
 %\author{José Mario López}
 %\date{\today}
%\maketitle
 
 \section{Introducción}
El propósito de cada práctica es afianzar conceptos y utilización de comandos de configuración de los temas abordados en las clases teóricas. El fin primordial es desarrollar las habilidades necesarias en comprensión y configuración de aspectos necesarios en el tema de redes de datos.\\

Para la realización de las prácticas se utilizará la versión 7.1.1 de Packet Tracer y el estudiante debe crear una cuenta en el sitio web de Cisco Networking Academy www.netacad.com
 \subsection{Estructura}
Las prácticas a desarrollar constan de 
\begin{description}
\item[Revisión previa (RP)] Lecturas extra-clase de la temática a desarrollar
\item[Parte obligatoria (PO)] Actividades mínimas de configuración de dispositivos y/o funcionalidades.
\item[Parte complementaria (PC)] Actividades que amplían las funcionalidades previamente configuradas y su objetivo principal es dotar de mejores características las topologías configuradas.
\item[Revisión posterior (RP)] Dirigido a evaluar los contenidos realizados en la práctica en cuestión, y así afianzar los conocimientos.
\end{description}
\vspace{10px}
Respecto de la calificación (6 puntos) o la distribución que el profesor estime conveniente si el valor de la práctica es diferente a 6 puntos, la distribución es la siguiente:
\begin{description}
\item[Revisión previa] 2 puntos oro
\item[Parte obligatoria] 1 puntos oro
\item[Parte opcional]  0.5 puntos oro
\item[Revisión posterior] 2.5 puntos oro
\end{description}

\subsection{Cronograma}
Para asegurar el mayor provecho de utilizar esta metodología, las actividades serán distribuidas como sigue:
\begin{description}
\item[1er hora-clase]  Revisión previa
\item[2da hora-clase] Desarrollo en clase de la temática correspondiente  
\item[3er hora-clase]  Desarrollo Parte obligatoria
\item[4ta hora-clase]  Desarrollo Parte opcional
\item[extra clase] Revisión posterior y asignación de tarea/ lecturas requeridas
\end{description}

\subsection{Reglamento}
\begin{enumerate}
\item Las actividades de revisión previa y posterior no serán postergables.
\item Las actividades obligatorias deben completarse en la tercer hora-clase y, de ser necesario, la cuarta hora-clase.
\item Para contar con la nota de la parte opcional, debe haberse finalizado la parte obligatoria en las horas de clase. Si a un estudiante le toma dos horas-clase terminar la parte práctica, puede realizar las actividades opcionales en horario extraclase.
\item Para dar por revisada una práctica, debe mostrarse al profesor el trabajo realizado y la topología debe tener conectividad total.
\item Una vez finalizada la práctica, el estudiante debe subir el archivo de Packet Tracer al enlace correspondiente en el campus virtual en la fecha establecida.
\end{enumerate}

\vfill
{\textbf {\normalsize José Mario López}}

{\small Profesor de la asignatura\\}

{\footnotesize 
De presentarse alguna duda respecto al contenido del documento, recuerde que puede abocarse al Departamento de Ing. En Sistemas, en hora de consulta, de lunes a viernes; o enviar un correo a jmlopezc@unah.edu.hn}
\end{document}