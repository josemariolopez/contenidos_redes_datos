\documentclass[12pt]{article}
\usepackage[utf8]{inputenc}
%\usepackage[T1]{fontenc}
%\usepackage{dejavu}

\usepackage{listings}
\usepackage{amsfonts}
\usepackage{fancyhdr}
\usepackage{comment}
\usepackage{graphicx}
\usepackage[letterpaper, top=2.5cm, bottom=2.5cm, left=2.2cm, right=2.2cm]{geometry}

\begin{document}

\begin{center}
%\includegraphics{logo_unah.png}\\
\begin{center}
%\includegraphics{logo_unah.png}\\
\bfseries{Universidad Nacional Autónoma de Honduras}\\
Facultad de Ingeniería\\
Departamento de Ingeniería en Sistemas\\
\bigskip
\bigskip
Asignatura: IS-611 Redes de Datos 2\\
Impartida  por José Mario López
\end{center}

\noindent\rule{\textwidth}{1pt}
\huge{VLAN e Introducción a VTP}
\noindent\rule{\textwidth}{1pt}
\end{center}
%\title{Resumen sobre Spanning-Tree Protocol}
 %\author{José Mario López}
 %\date{\today}
%\maketitle
 
\section{Introducción}   
La utilización de VLAN contribuye, sin duda, a gestionar mejor la infraestructura de red, pues son varios los beneficios que obtenemos al configuralas, sin embargo, a medida que aumenta el número de switches en la red de empresas pequeñas o medianas, la tarea de administrar las VLAN y los enlaces troncales en una red se vuelve un desafío, y es allí donde entra en juego el protocolo VTP.

\section{VTP - VLAN Trunk Protocol}

\textbf{VTP} es un protocolo propietario de Cisco, que está disponible en la mayoría de los equipos de la serie Catalyst, que reduce la administración de VLAN en una red conmutada.

\paragraph{Beneficios del VTP}\\
\begin{itemize}
    \item Consistencia en la configuración de VLAN a través de la red
    \item Seguimiento y monitoreo preciso de las VLAN
    \item Informes dinámicos sobre las VLAN que se agregan a la red
    \item Configuración de enlace troncal dinámico cuando las VLAN se agregan a la red
\end{itemize}

 \subsection{Conceptos de VTP}
  \subsubsection{Publicaciones del VTP}
VTP usa una jerarquía de publicaciones para distribuir y sincronizar las configuraciones de la VLAN a través de la red; Distribuyen el nombre de dominio del VTP y cambios en la 
configuración de la VLAN a los switches habilitados por 
el VTP.\\

Entre las publicaciones del VTP se encuentran:
\begin{description}
\item[Publicaciones de resumen]{La publicación del resumen contiene el nombre de dominio del 
VTP, el número actual de revisión y otros detalles de la 
configuración del VTP.}
\item[Publicaciones de subconjunto]{Una publicación de subconjunto contiene información de la VLAN.}
\item[Publicaciones de solicitud]{Cuando una publicación de solicitud se envía al servidor del VTP 
en el mismo dominio del VTP, el servidor del VTP responde con el 
envío de una publicación del resumen y luego una publicación de 
subconjunto.}
\end{description}

\subsubsection{Encapsulamiento de la trama del VTP}
Una trama del VTP consiste en un campo de encabezado y un 
campo de mensaje. La información del VTP se inserta en el 
campo de datos de una trama de Ethernet. La trama de Ethernet 
luego se encapsula como una trama troncal de 802.1Q (o trama 
ISL). Cada switch en el dominio envía publicaciones periódicas de 
cada puerto de enlace troncal a una dirección multicast reservada.
Los switches vecinos reciben estas publicaciones que actualizan 
las configuraciones de sus VTP y VLAN si es necesario.

Las tramas del VTP contienen la siguiente información para 
cada VLAN configurada: 
\begin{itemize}
    \item ID de la VLAN (IEEE 802.1Q)
    \item Nombre de la VLAN
    \item Tipo de la VLAN
    \item Estado de la VLAN
    \item Información adicional de configuración de la VLAN específica para el tipo de VLAN 
\end{itemize}
Varios de estos campos de información no son de interés para el desarrollo del curso, y quedará a discreción del estudiante su posterior investigación.

\subsubsection{Número de revisión del VTP}
El número de revisión de la configuración es un número de 32 bits 
que indica el nivel de revisión para una trama del VTP. El número 
de configuración predeterminado para un switch es cero. Cada 
vez que se agrega o elimina una VLAN, se aumenta el número de 
revisión de la configuración. Cada dispositivo de VTP rastrea el 
número de revisión de configuración del VTP que se le asigna.



 \subsubsection{Dominio de VTP}
Consiste en uno o más switches interconectados. Todos los switches en un dominio comparten los detalles de configuración de la VLAN con las publicaciones del VTP. \textbf{Un router o switch de Capa 3 define el límite de cada dominio}.\\

Un dominio VTP se identifica por un nombre y adicionalmente, puede configurarse una contraseña VTP para que el switch pueda formar parte del dominio, dicha contraseña debe ser igual en todos los switches, sino, no podrá establecerse la conectividad.

\subsubsection{Modos de operación de VTP}
 \begin{description}
 \item[Servidor VTP] Los servidores del VTP publican la información VLAN del dominio del VTP a otros switches habilitados por VTP en el mismo dominio VTP. Los servidores VTP guardan la información de la VLAN para el dominio completo en la NVRAM. El servidor es donde la VLAN puede ser creada, eliminada o redenominada para el dominio.
 \item[Cliente del VTP] Los clientes VTP funcionan de la misma manera que los servidores VTP pero no se puede crear, cambiar ni eliminar las VLAN en un cliente VTP. Un cliente VTP sólo guarda la información de la VLAN para el dominio completo mientras el switch está activado. Un reinicio del switch borra la información de la VLAN. Para indicar que un switch funciona en modo cliente, debe configurarlo en ese modo explícitamente. Es importante mencionar, que de manera predeterminada los switches están en modo Servidor.
 \item[VTP transparente] Los switches en modo transparente reenvían publicaciones del VTP a los clientes VTP y servidores VTP; pero no participan de las actualizaciones de VTP. Las VLAN que se crean, redenominan o se eliminan en los switches transparentes son locales a ese switch solamente.
 \end{description}
\subsection{Configuración de VTP}
\subsubsection{En el Servidor VTP}
\begin{enumerate}
    \item Confirme las configuraciones predeterminadas
    \item Configure el dominio
    \item Configure la contraseña para el dominio
    \item Asegúrese que todos los switches estén con la misma versión de VTP [para efectos de la Clase, será la v2]
    \item Configure las VLAN
    \item Configure los puertos troncales
\end{enumerate}

\subsubsection{En el Cliente VTP}
\begin{enumerate}
    \item Confirme las configuraciones predeterminadas
    \item Configure el modo Cliente de VTP
    \item Configure el dominio
    \item Configure la contraseña para el dominio
    \item Asegúrese que todos los switches estén con la misma versión de VTP [para efectos de la Clase, será la v2]
    \item Conecte con el servidor VTP (o los servidores)
    \item Configure los puertos troncales
\end{enumerate}

%\begin{lstlisting}
%Router(config)\# router rip
%Router(config-router)\# network \textit{dirección de red}
%Router(config-router)\# version \textit{versión}
%Router(config-router)\# maximun-paths \textit{número}
%Router(config-router)\# no auto-summary
%Router(config-router)\# redistribute static
%\end{lstlisting}
%\vspace{5px}

\vfill
{\textbf {\normalsize José Mario López}}

{\small Profesor de la asignatura\\}

{\footnotesize 
De presentarse alguna duda respecto al contenido del documento, recuerde que puede abocarse al Departamento de Ing. En Sistemas, en hora de consulta, de lunes a viernes; o enviar un correo a jmlopezc@unah.edu.hn}
\end{document}